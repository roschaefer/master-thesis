\documentclass{article}
\begin{document}

\title{Here could be a title}\label{here-could-be-a-title}
\author{Robert Schäfer\\ Department of Computer Graphics, Hasso-Plattner-Institut}
\maketitle

\newcommand{\rufu}{Rundfunk \textsc{mitbestimmen}}

\begin{abstract}
Coordinated multiple views are a well known area of research and it has
been shown that they create better insights.
However, coordinated multiple views tend to become increasingly complex.
In our use case we introduce a version of coordinated multiple views that allows for different levels of detail and therefore targets both experts and lay people.
\end{abstract}

\section{Introduction}
In many cases, insights based on visualization techniques like coordinated multiple views are used for strategic decision-making and are made for experts.
Thus, most data visualizations are made exclusively for professionals.
There is, however, a big popularity of data visualizations in the media.
E.g. data journalism is the fastest growing segment in journalism.
We believe that advanced data visualization techniques like coordinated multiple views can be successfully transferred in to a public environment.

\section{Research}


Jan Kriesel states that ``there is only one broadband connection to the
brain: The eyes``.

\section{Use case}

We use data visualizations in the context of the tool \rufu{}.
\rufu{} is a web application were payers of German public broadcasting fees can publish how their broadcasting fees should be spent.
Since 2013 these fees are compulsory for every home in Germany and as a consequence, broadcasting receives €8,000,000,000 annually.
Yet it is subject to little or no public feedback, ranking, or even debate on what constitutes value or quality.
With \rufu{} we see a win-win situation where payers of the fees and broadcasters can benefit from each other.
Broadcasters can evaluate their program which in turn creates a public feedback for payers of the broadcasting fees.

Considering data visualizations we are facing two challenges in this scenario:
\begin{enumerate}
\item
Visualisations must be easy to understand and intuitive to interact with
\item
Visualizations must be detailed enough to evaluate the program
\end{enumerate}

\end{document}


%Our use case is creating a tool to publicly evaluate public broadcasting in Germany.

%Public broadcasting has a long history in all of Europe.
%Especially after the experiences of the Third Reich, freedom of the press and freedom of broadcasting in particular was defined in the German constitution, the basic law.
%Article 5 not only ensures the press to be free of censorship.
%It is interpreted in such a way that it guarantees broadcasting to exist and be politically and economically independent.
%It is remarkably in many ways:
%First, mass media are recognised to be a basic prerequisite for formation of opinion of the general public.
%Second, the state fosters free access to information, ie. Open Knowledge.

%Despite this highly positive ideal, reality looks somewhat different.
%Public broadcasting in Germany receives €8,000,000,000 (eight billion euros) annually, yet it is subject to little or no public feedback, ranking, or even debate on what constitutes value or quality.
%Since 2013, there is no legal opt-out for German citizen anymore.
%Every home in Germany has to pay for broadcasting whether or not the people actually use it.
%This has created numerous constitutional complaints and approximately 2 million homes in Germany refuse to pay, even it is virtually illegal to do so.

