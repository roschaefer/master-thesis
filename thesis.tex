\documentclass{article}
\usepackage{todonotes}
\usepackage{biblatex}
\usepackage{listings}
\usepackage{color}
\usepackage{copyrightbox}
\usepackage{ccicons}
\definecolor{lightgray}{rgb}{.9,.9,.9}
\definecolor{darkgray}{rgb}{.4,.4,.4}
\definecolor{purple}{rgb}{0.65, 0.12, 0.82}
\lstdefinelanguage{JavaScript}{
  keywords={break, case, catch, continue, debugger, default, delete, do, else, false, finally, for, function, if, in, instanceof, new, null, return, switch, this, throw, true, try, typeof, var, void, while, with},
  morecomment=[l]{//},
  morecomment=[s]{/*}{*/},
  morestring=[b]',
  morestring=[b]",
  ndkeywords={class, export, boolean, throw, implements, import, this},
  keywordstyle=\color{blue}\bfseries,
  ndkeywordstyle=\color{darkgray}\bfseries,
  identifierstyle=\color{black},
  commentstyle=\color{purple}\ttfamily,
  stringstyle=\color{red}\ttfamily,
  sensitive=true
}

\lstset{
   language=JavaScript,
   backgroundcolor=\color{lightgray},
   extendedchars=true,
   basicstyle=\footnotesize\ttfamily,
   showstringspaces=false,
   showspaces=false,
   numbers=left,
   numberstyle=\footnotesize,
   numbersep=9pt,
   tabsize=2,
   breaklines=true,
   showtabs=false,
   captionpos=b
}
\bibliography{bibliography}

\begin{document}

\title{Multiple coordinated views of massive geo data using tree maps and choropleth maps}
\author{Robert Schäfer\\ Department of Computer Graphics, Hasso-Plattner-Institut}
\maketitle

\newcommand{\rufu}{``Rundfunk \textsc{MITBESTIMMEN}''}
\newcommand\hmm[1]{\ifnum\ifhmode\spacefactor\else2000\fi>1000 \uppercase{#1}\else#1\fi}
\newcommand{\cmv}{\hmm{c}oordinated multiple view}
\newcommand{\cmvs}{\hmm{c}oordinated multiple views}
\newcommand{\map}{2D Map}
\newcommand{\maps}{2D Maps}
\newcommand{\tmap}{2.5 Tree Map}
\newcommand{\tmaps}{2.5 Tree Maps}
\newcommand{\dss}{\hmm{d}ecision support systems}

\begin{abstract}
  Numerous visualization techniques exist for data-driven \dss{}.
  Two of which, namely tree maps and choropleth maps are especially useful when dealing with hierarchical and geographical data respectively.
  While a choropleth map is an sensible choice for geographical data, there is no obvious mapping of arbitrary hierarchical data.
  On the other hand, tree maps allow to visualize arbitrary, hierarchical and multidimensional data using nested rectangles.
  Although a tree map looks similar to a geographical map, the result may not be as intuitive and comprehensible.
  There is no predefined layouting for arbitrary data, let alone no well known layouting that everybody feels comfortable with.
  In this thesis, we propose and evaluate a coordinated multiple view to combine advantages of both visualizations.
  In one view, the user gets an easy access with a familiar geographical map, while in the other view the same data is displayed in a tree map.
  \todo[inline]{Should we mention the combination of multiple data sets here?}
\end{abstract}

\section{Introduction}
The human brain processes visual information better than it processes text.
As a result, the most tangible data analyses usually come with some sort of data visualization.
Data visualizations on a computer allow for user interactions and different levels of granularity according to the customer demands, which provides a great user experience.
There are a multitude of different techniques.
\cmvs{} take data visualizations to the next level by exploiting the respective advantages of the used visualizations as much as possible.
This combination may yield a greater value to the user, but it is unclear what kind visualization techniques work best together and for which kind of data.
How a user interact with \cmvs{} and how it differs from the use of single visualizations is another question worth to investigate.
In this thesis, we consider the question how a combination of a geographical visualization with a hierarchical visualization performs, for what kind of data this combination is suitable and what interaction patterns apply.

% Three moves
%\todo{What is the topic about?}
\todo[inline]{Establish the niche, why is there further research on your topic?}
\todo[inline]{Introduce the current research, what's the hypothesis, the research question?}

\subsection{Motivation}

We create data visualizations of multi-dimensional, hierarchical and geographical data.
Namely, we develop \rufu{} which is an application for German citizen to publish which public broadcasts should benefit from their broadcasting fees.
The output of this application is a public user ranking and it can be used by broadcasting corporations to evaluate their program.
Data visualizations guide media researchers, journalists and the general audience to draw conclusions.
In this particular use case, the selection and interaction with the data may happen geographically, but the desired visualization could show e.g. changes over time or relationships within the data.

To explain this a little deeper:
Public broadcasting in Germany is organized federally, a German home belongs to the jurisdiction of a public broadcasting corporation.
But the produced content can be used all German citizen and it is even required to be free and available to everyone.
So this means that e.g. a media researcher might want to select all users from within a certain region, but is actually interested into relationships of broadcasts that are preferred by people from that area.

So the interaction and selection of data should happen in another view than the actual data visualization.
Since we deal with geographical data, we use geometry on a map for selection and interaction.
For the visualization, we use a different technique, e.g. a tree map if we deal with hierarchical data.

\subsection{Problem statement}

% \todo[inline]{Start off with the problem of tree map vs geo map}

Tree maps visualize hierarchical data on a two dimensional canvas and are particularly suitable if the proportions of the data should be emphasized.
When dealing with both multidimensional and geographical data, problems arise when other features than geographical features are used for the layouting of the tree map.
As the order and placement of items depend on their specific values and hierarchy, items that should belong together according to their geographic circumstances may be scattered across the tree map.
This obstructs the comprehensibility of tree maps and makes it especially difficult to select geographical units of items.


\subsection{Research questions}

When dealing with multi-dimensional data, is it helpful to have multiple views for these dimensions?
What are best practices for the implementation of \cmvs{}?
Is it great user experience to have one view for interaction and another view for the data visualization to create insights?
Do people prefer one view for the interaction, e.g. the geographical dimension, or do they use all views for interaction and visualization alternately?

\subsection{Objectives}

\todo[inline]{No section without text}

\paragraph{Literature research and implementation overview}
In this paper we will give an overview on \cmvs{} and focus on visual analytics as well as massive, geographical data.
Existing concepts and implementations of \cmvs{} are examined and summarized in an overview.

\paragraph{Development of a basic \cmv{}}
The existing Visual Analytics platform is supplemented with a basic \cmv{} system.
Modules, interfaces and functionalities of the \cmv{} system are designed, coordinated and prototypically implemented.
In particular, a method for arranging multiple \cmv{} widgets in a \cmv{} layout and storing \cmv{} layouts is developed.
\todo[inline]{This is translated from the draft. Honestly I don't understand what this part is about}

\paragraph{Development of a brushing feature}
We develop coupling mechanisms and interaction mechanisms between \maps{} (for map-based representation) and \tmaps{} (for abstract information representation).
The functionality includes zoom per object or selection with a bounding box.
This creates a powerful selection mechanism, which can be used to select the data from the map-based representation in the \tmap{}.

\paragraph{Demonstration and evaluation}
\cmv{} layouts and suitable views are implemented and tested for the selected test data.
Based on the test data sets, the \cmv{} implementation is examined and evaluated for design criteria\cite{Baldonado2000}, general usability aspects\cite{Roberts2007} and usage for typical Visual Analytics tasks.

% Our goal is to create meaningful data.
% This includes to encourage as many people as possible to publish their interests.
% It also involves to provide access to information about the preferred broadcasts.
% We suggest that users actively publish data and passively examine the summary of data of all users.
% Broadcasters on the other side receive the data and evaluate the program and actively change the program according to the interest of the audience.
% \todo[inline]{explain feedback loop}
% The mentioned data visualization need to be interactive to show the user the desired level of detail.

%\subsubsection{Use case specific goals}
%We propose the hypothesis that people behave differently when they decide consciously and when they decide with their remote control.
%The hypothesis goes even further, explicit interests may fit to the programme mandate of public broadcasting in contrast to TV and radio ratings which are based on usage data.

%As the data is usage independent we want to gain knowledge about the entire population including those who don't use broadcasting at all.
%In this manner we want to tackle the mentioned overfitting problem.

\subsection{Methodological approach}

In our work on the project \rufu{} we design and prototype visualizations for a very specific use case.
We fully control the database and the database schema as well as on the user facing application on top of it.
At the same time, we are developing a general purpose visual analytics platform for geographical and hierarchical data visualizations.

For the scope of this master thesis we therefore compare implemented visualization and views.
How do both approaches differ in development speed, value for the customer?
What considerations need to be done regarding the database schema?

\section{Structure of the work}
\todo[inline]{Got any structure?}


\newpage

\section{State of the art}
Data visualizations are a key part in data-driven \dss{}\cite{Nada2007}\cite{Poleto2015}.
\textcite{Few2013} mentions sense-making (also called data analysis) and communication as some of the most important purposes of data visualization.
Statistical information is abstract and in data visualization ``we must find a way to give form to that which has none.''\cite{Few2013}

Visualizations are an obvious choice for managers who demand a quick overview on performance data.
In fact \textcite{Kusinitz2014} explains that the human brain processes visual information 60,000 times faster than text and visual content makes up even 93\% of all human communication.
Data visualizations are essential here, as managers often do not have the resources to do an in depth analysis with the numbers only.
We can expect to see these technologies more in more in business applications.
\textcite{McAfee2012} from the MIT Center of Digital Business showed that organizations driven most by data-based decision making had 4\% higher productivity rates and 6\% higher profits.
However, little research has been done regarding the performance of \cmvs{} in the field of decision making.
There might be a great potential.
Back in 1997 \textcite{Mayer1997} conducted eight studies to compare the effect of using multimedia on university students.
The studies showed that when using combined visual and verbal explanations the generation of creative problem solutions increased by an average of more than 50\%.

So the application of combined data visualization techniques in decision making seems to be a promising strategy.
Nevertheless is is unclear, which visualization techniques are the most suitable to be used in combination.
If we know what kind of data we are dealing with, what are the best suited visualization techniques?
Let's say we have multidimensional data, is there an order in how people access these multiple dimensions?
How do these visualizations perform and what are best practices to be considered for their implementation?




\subsection{Visualization of hierarchical data}
Due to the hierarchical nature of the of our use cases, we focus on the visualization of hierarchical and geographical data
The visualization of hierarchical data has a long tradition.
The traditional representation of a tree is a rooted, directed graph with the root node at the top.
An everyday use case is a directory tree example of a file system, e.g. in file browsers or command line utilities like \texttt{tree} in UNIX.
As \textcite{Shneiderman1992} mentions, this visualization becomes increasingly large when displaying more than one node and soon exceeds the entire screen size.
\textcite{Johnson1991} proposes the tree map visualization technique, in which each node is a rectangle whose area is proportional to some attribute, thus making 100\% use of the available screen size.
As we can see in figure~\ref{fig:research:treemap} large boxes are labeled with generic tems like ``cars'' and ``medicaments'' and include smaller boxes with more specific meanings.
We apply the same rules to ordinary maps.
The world can be divided into continents, which can be divided into countries, which can be divided into provinces and so on.
The difference is that there is no predefined algorithm for layouting, which brings up one of the major disadvantages of tree maps:
As the order of placement depends on the respective features of the nodes, small changes in the input data can lead to a large change in the layout of the resulting tree map.
%\todo{What are tree maps?}

\begin{figure}[h]
\centering
  \copyrightbox[b]{\includegraphics[width=\textwidth]{images/treemap_example}}{ \hfill \ccAttribution \ccShareAlike \hspace{1mm} Observatory of Economic Complexity\cite{Macro2017}}
\caption{German exports visualized as a tree map}
\label{fig:research:treemap}
\end{figure}

In 2004, \textcite{Bladh2004} transfer the concept of tree maps from two dimensional into three dimensional space.
The introduce StepTree, which is a three dimensional tree map to display a directory layout.
It ``differs from Treemap in that it employs three dimensions by stacking each subdirectory on top of its parent directory.''\cite{Bladh2004}
3D tree maps are superior to 2D tree maps for tasks with a pronounced topological challenge.
User perform significantly better in interpreting the hierarchical structure.
Yet, 3D visualizations come with some disadvantages as superimposition of objects and a complex view point navigation.
We can see an example of a 3D tree map in figure~\ref{fig:research:ua_treemap}

\begin{figure}[h]
\centering
  \includegraphics[width=\textwidth]{images/ua_treemap_example}
  \caption{User distribution of \rufu{} across German federal states}
\label{fig:research:ua_treemap}
\end{figure}

\todo[inline]{What's the difference of 2.5D tree maps and 3D tree maps?}

\subsection{Choropleth map}
A choropleth map is a thematic map in which areas are shaded or patterned in proportion to the measurement of the statistical variable being displayed on the map.
A popular use case is the display of population density or per-capita income.
We can see an example of a choropleth map in figure~\ref{fig:research:choropleth}.
Choropleth maps are extremely popular and so the audience is likely to understand them.
They are very helpful when data is attached to enumeration unites like counties, provinces and countries.

\begin{figure}[h]
\centering
  \copyrightbox[b]{\includegraphics[width=\textwidth]{images/choropleth_example}}{GPLv3 Mike Bostock\cite{Bostock2017:2}}
  \caption{Unemployment rate in the USA}
\label{fig:research:choropleth}
\end{figure}

\todo[inline]{How do tree maps relate to geographical data?}

\subsection{Coordinated multiple views}
According to \textcite{Roberts2007} \cmvs{} is just ``a specific exploratory visualization technique that enables users to explore their data''.
\cmvs{} are characterized by the fact, that they show multiple views side-by-side.
Most multiple coordinated views also provide some kind of brushing technique.
``The technique of brushing is the principle approach, where elements are selected (and highlighted) in one display, concurrently the same information in any other linked display is also highlighted.''\cite{Roberts2007}
%\todo{What are CMVs?}
We can see an example in figure~\ref{fig:research:cmv}.
It displays an on-time performance of airlines, visualized with the ``Crossfilter'' javascript library.
The user can set the borders of an interval with the mouse in each of the views.
The visualization takes the most recent 80 flights from the database that match all given filters.
All visualizations are then updated in real time.
As we can see in the example in figure~\ref{fig:research:cmv} there seems to be a correlation of a long delay with a later time of the day.

\begin{figure}[h]
\centering
  \copyrightbox[b]{\includegraphics[width=\textwidth]{images/cmv_example}}{Crossfilter\cite{Bostock2017}}
\caption{Airline on-time performance: Correlation of time of day with arrival delay. Most recent flight with a delay of more than 100 minutes selected.}
\label{fig:research:cmv}
\end{figure}

\section{Methodology}

Since we deal with real world problems, we aim to evaluate the developed tools on real users and existing data.

\subsection{Data sets}
For our uses cases, we have two different data sets:
The first one, called ``RISO'', consists of statistical data from various German administrations and is used by the authorities for urban planning and policy strategies.
The other set of data consists of a user ranking of public broadcasting in Germany, i.e. entities, mostly TV and radio broadcasts, are liked or disliked by people.
The latter is public data and can be used by media researchers of broadcasting corporations but also targets media journalists and the general audience.

Both data sets share some characteristics.
The administrative data connects certain features with certain regions of Germany.
As Germany is a federal state, larger regions consist of many other smaller regions.

The second one consists of user data that was collected through a web application called ``Rundfunk MITBESTIMMEN''.

\todo[inline]{Explain the two different data sets}

\subsubsection{RISO}

The RISO data base is used in by local authorities to get insights about governmental KPIs to assist local and regional decision making.
It is a relational database with the three most important tables listed in figure~\ref{fig:data:riso}.
\begin{figure}[h]
\centering
  \includegraphics[width=\textwidth]{images/riso}
  \caption{The three most important tables of the RISO database}
  \label{fig:data:riso}
\end{figure}

The largest table is called ``data'', with approximately 10,466,600 records, which holds all values along with the survey date.
\paragraph{Features}
This data is connected to a feature table through a foreign key called ``id\_mm''.
In the feature table we can find the description for every referenced feature, e.g. population density, working population in agriculture, education spending.
The ``RISO'' system groups all features in a 4-level hierarchy:
\begin{enumerate}
  \item
    katalog\_daten\_1\_kategorien
  \item
    katalog\_daten\_2\_themenfelder
  \item
    katalog\_daten\_3\_themen
  \item
    katalog\_daten\_4\_merkmale
\end{enumerate}
The actual features table is the last one in the list.
As it is the lowest level within the hierarchy, this features table holds the most records (1234 records).


\paragraph{Regions}
On the other side, the geometry data is stored in the ``regions'' table and in particular the ``geojson'' column.
The ``data'' and ``regions'' tables are, again, connected via the foreign keys ``id\_re'' and ``regs''.
Unlike the feature table, the regions are grouped through the ``id\_re'' that indicates the hierarchy level.
This is a foreign key to a table ``katalog\_raumebenen'', so e.g. a region with a ``id\_re'' of 1 is a federal state of Germany.
Both column ``id\_re'' and ``regs'' belong to the primary key of the regions table, so there will never be two regions on the same hierarchy level with the same ``regs'' id.


The schema of the ``RISO'' database allows to add data of arbitrary size, features and completeness as long as we can store continuous values along withe geometry data.


\subsubsection{Rundfunk MITBESTIMMEN}
Unlike the ``RISO'' database, the schema of ``Rundfunk MITBESTIMMEN'' follows the requirements of a productive web application.
This web application allows user to vote on broadcasts.
Figure~\ref{fig:data:rundfunk} shows how user data is connected to entities, i.e. broadcasts in our use case.
Consequently, the ``users'' table is connected to the ``broadcasts'' table via the ``selections'' table.
The selections table stores the response of a user to the broadcast, e.g. ``positive'' or ``neutral'', as well as an amount in Euro in case the user wants to allocate a share of the monhtly budget to that broadcast.

\begin{figure}[h]
\centering
  \includegraphics[width=\textwidth]{images/er}
  \caption{Database schema of Rundfunk MITBESTIMMEN app}
  \label{fig:data:rundfunk}
\end{figure}

On both the users table and the broadcast table, we have additional data.
Every user has a \texttt{latitude}, \texttt{longitude} as well as a \texttt{city}, \texttt{postal\_code} and a \texttt{state\_code}.
The broadcast in turn is connected to a \texttt{Station} and also has a \texttt{Medium}.
A station could be a tv or a radio station.
The medium of a broadcast can be one of the following: \texttt{TV}, \texttt{Radio}, \texttt{Online} or \texttt{Other}.
\\
\\
With this data, we can reason about the following questions:
\begin{enumerate}
  \item
    Which broadcasts are selected together very often?
  \item
    Which groups of broadcasts, e.g. all broadcasts of a specific TV station, are performing very well?
  \item
    Wich broadcasts are selected in wich region of Germany?
\end{enumerate}

For the latter, we aggregate the user data with regions and output a geojson file that can be used in our data visualization tool.
Listing~\ref{lst:geojson:example} shows an example.
\lstinputlisting[language=JavaScript, label={lst:geojson:example}, caption={Geojson example}]{listings/example.geojson}


To achieve that we develop a web application where payers of German public broadcasting fees can publish how their broadcasting fees should be spent.
This tool is called \rufu{}.
The workflow goes like this:
\begin{enumerate}
\item Users search for broadcasts
\item Users select broadcasts they want to support
\item Users distribute a virtual budget among the selected broadcasts
\end{enumerate}
\todo[inline]{explain tool in depth}

Considering data visualizations we are facing two challenges in this scenario:
\begin{enumerate}
\item
Visualisations must be easy to understand and intuitive to interact with
\item
Visualizations must be detailed enough to evaluate the program
\end{enumerate}

Make the website user friendly, e.g. guide users as quick as possible to broadcasts they want to support.

\todo[inline]{What did work?}
\todo[inline]{What didn't work?}
\todo[inline]{Current drawbacks}
\todo[inline]{What's missing}




\clearpage
\printbibliography
\end{document}



%\subsubsection{Use case specific problems}
%There is striking discrepancy of intentions between available metrics and the programme mandate of public broadcasting.
%TV ratings are produced by a group of 5000 representative homes equipped with a special TV box, while Radio ratings are carried out by phone surveys over the course of a year.
%Although created differently they share the same intention, i.e.\ to sell advertisements.
%
%We suspect that public broadcasting suffers from an overfitting problem:
%Since broadcasters have usage data only, they focus too much on people who still listen to the radio and watch TV.
%Young people especially show a decreasing interest in conventional mass media which has been shown in surveys.
%Public broadcasting fails in that respect that everybody has to pay and thus has a right to get access to information.

%In many cases, insights based on visualization techniques like \cmvs{} are used by experts for strategic decision-making.
%Thus, many advanced data visualizations techniques are made exclusively for professionals.
%
%On the other hand, data visualizations widely popular:
%Data journalism is one of the emerging fields in journalism.
%Facts and figures are the strongest evidence for opinionated journalistic reports.
%Not even a football match can do without a number based analysis.
%
%We believe that advanced data visualization techniques can be adopted and used in both an informative and strategic way by professionals as well as lay people.



% We use data visualizations in the context of the expenditure of public broadcasting fees in Germany.
% Since the year 2013 these fees are compulsory for every home in Germany and as a consequence, broadcasting receives €8,000,000,000 annually.
% Yet it is subject to little or no public feedback, ranking, or even debate on what constitutes value or quality.
% There is neither transparency on how the fees are spent nor a public feedback on how the fees should be spent.
% 
% We see a great potential, a win-win situation to be precise, because both payers of the broadcasting fees and broadcasters themselves can benefit from each other.
% As of 2013 there is no legal opt-out anymore and people have a strong interest to say how their fees should be spent.
% Broadcasters on the other hand can evaluate their program based on the interests delivered by the users.
% This reciprocal relationship would create a public feedback for payers and a better program for broadcasters.
% 
% This problem is perfectly suitable to be tackled with data visualizations.
% \todo[inline]{Explain why data visualization are good for communication}

%Our use case is creating a tool to publicly evaluate public broadcasting in Germany.

%Public broadcasting has a long history in all of Europe.
%Especially after the experiences of the Third Reich, freedom of the press and freedom of broadcasting in particular was defined in the German constitution, the basic law.
%Article 5 not only ensures the press to be free of censorship.
%It is interpreted in such a way that it guarantees broadcasting to exist and be politically and economically independent.
%It is remarkably in many ways:
%First, mass media are recognised to be a basic prerequisite for formation of opinion of the general public.
%Second, the state fosters free access to information, ie. Open Knowledge.

%Despite this highly positive ideal, reality looks somewhat different.
%Public broadcasting in Germany receives €8,000,000,000 (eight billion euros) annually, yet it is subject to little or no public feedback, ranking, or even debate on what constitutes value or quality.
%Since 2013, there is no legal opt-out for German citizen anymore.
%Every home in Germany has to pay for broadcasting whether or not the people actually use it.
%This has created numerous constitutional complaints and approximately 2 million homes in Germany refuse to pay, even it is virtually illegal to do so.

