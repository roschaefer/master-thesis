\documentclass{article}
\usepackage{todonotes}

\begin{document}

\title{Here could be a title}\label{here-could-be-a-title}
\author{Robert Schäfer\\ Department of Computer Graphics, Hasso-Plattner-Institut}
\maketitle

\newcommand{\rufu}{Rundfunk \textsc{mitbestimmen}}

\begin{abstract}
Coordinated multiple views are a well known area of research and it has
been shown that they create better insights.
However, coordinated multiple views tend to become increasingly complex.
In our use case we introduce a version of coordinated multiple views that allows for different levels of detail and therefore targets both experts and lay people.
\end{abstract}

\section{Introduction}

\subsection{Motivation}

In many cases, insights based on visualization techniques like coordinated multiple views are used by experts for strategic decision-making.
Thus, many advanced data visualizations techniques are made exclusively for professionals.

On the other hand, data visualizations widely popular:
Data journalism is one of the emerging fields in journalism.
Facts and figures are the strongest evidence for opinionated journalistic reports.
Not even a football match can do without a number based analysis.

We believe that advanced data visualization techniques can be adopted and used in both an informative and strategic way by professionals as well as lay people.

\subsection{Problem statement}

We use data visualizations in the context of the expenditure of public broadcasting fees in Germany.
Since the year 2013 these fees are compulsory for every home in Germany and as a consequence, broadcasting receives €8,000,000,000 annually.
Yet it is subject to little or no public feedback, ranking, or even debate on what constitutes value or quality.
There is neither transparency on how the fees are spent nor a public feedback on how the fees should be spent.

We see a great potential, a win-win situation to be precise, because both payers of the broadcasting fees and broadcasters themselves can benefit from each other.
As of 2013 there is no legal opt-out anymore and people have a strong interest to say how their fees should be spent.
Broadcasters on the other hand can evaluate their program based on the interests delivered by the users.
This reciprocal relationship would create a public feedback for payers and a better program for broadcasters.

This problem is perfectly suitable to be tackled with data visualizations.
\todo[inline]{Explain why data visualization are good for communication}

\subsubsection{Use case specific problems}
There is striking discrepancy of intentions between available metrics and the programme mandate of public broadcasting.
TV ratings are produced by a group of 5000 representative homes equipped with a special TV box, while Radio ratings are carried out by phone surveys over the course of a year.
Although created differently they share the same intention, i.e.\ to sell advertisements. 

We suspect that public broadcasting suffers from an overfitting problem:
Since broadcasters have usage data only, they focus too much on people who still listen to the radio and watch TV.
Young people especially show a decreasing interest in conventional mass media which has been shown in surveys.
Public broadcasting fails in that respect that everybody has to pay and thus has a right to get access to information.


\subsection{Aim of the work}
Our goal is to create meaningful data.
This includes to encourage as many people as possible to publish their interests.
It also involves to provide access to information about the preferred broadcasts.
We suggest that users actively publish data and passively examine the summary of data of all users.
Broadcasters on the other side receive the data and evaluate the program and actively change the program according to the interest of the audience.
\todo[inline]{explain feedback loop}
The mentioned data visualization need to be interactive to show the user the desired level of detail.

\subsubsection{Use case specific goals}
We propose the hypothesis that people behave differently when they decide consciously and when they decide with their remote control.
The hypothesis goes even further, explicit interests may fit to the programme mandate of public broadcasting in contrast to TV and radio ratings which are based on usage data.

As the data is usage independent we want to gain knowledge about the entire population including those who don't use broadcasting at all.
In this manner we want to tackle the mentioned overfitting problem.

\subsection{Methodological approach}
Our methodology is to collect real data from existing users.
To achieve that we develop a web application where payers of German public broadcasting fees can publish how their broadcasting fees should be spent.
This tool is called \rufu{}.
The workflow goes like this:
\begin{enumerate}
\item Users search for broadcasts
\item Users select broadcasts they want to support
\item Users distribute a virtual budget among the selected broadcasts
\end{enumerate}
\todo[inline]{explain tool in depth}

Considering data visualizations we are facing two challenges in this scenario:
\begin{enumerate}
\item
Visualisations must be easy to understand and intuitive to interact with
\item
Visualizations must be detailed enough to evaluate the program
\end{enumerate}

Make the website user friendly, e.g. guide users as quick as possible to broadcasts they want to support.

\section{State of the art}

\todo[inline]{What are combined multiple views?}
\todo[inline]{Current drawbacks}

Jan Kriesel states that ``there is only one broadband connection to the
brain: The eyes``.



\end{document}


%Our use case is creating a tool to publicly evaluate public broadcasting in Germany.

%Public broadcasting has a long history in all of Europe.
%Especially after the experiences of the Third Reich, freedom of the press and freedom of broadcasting in particular was defined in the German constitution, the basic law.
%Article 5 not only ensures the press to be free of censorship.
%It is interpreted in such a way that it guarantees broadcasting to exist and be politically and economically independent.
%It is remarkably in many ways:
%First, mass media are recognised to be a basic prerequisite for formation of opinion of the general public.
%Second, the state fosters free access to information, ie. Open Knowledge.

%Despite this highly positive ideal, reality looks somewhat different.
%Public broadcasting in Germany receives €8,000,000,000 (eight billion euros) annually, yet it is subject to little or no public feedback, ranking, or even debate on what constitutes value or quality.
%Since 2013, there is no legal opt-out for German citizen anymore.
%Every home in Germany has to pay for broadcasting whether or not the people actually use it.
%This has created numerous constitutional complaints and approximately 2 million homes in Germany refuse to pay, even it is virtually illegal to do so.

