\chapter*{Abstract}

Coordinated, multiple information visualizations can offer added value by combining the strengths of individual visualization techniques or data representations and by reflecting user interactions among multiple views simultaneously.
Specifically, when visualizing multi-dimensional and hierarchical geodata, the combination of digital maps and treemap-based visualizations can facilitate the comprehensibility and interactivity.
E.g. a linked geographic map can establish the geographical context if the used treemap layout algorithm does not take into account the spatial data.
In addition, the user can select elements as desired based on the geographic proximity in the map or based on the technical context in the treemap.

This master thesis provides the design and implementation of a coordinated visualization of a treemap and a geographic visualization.
The focus is on coordinated interactions, the information which is exchanged between individual views and the common intersection between data visualizations.

Based on existing scientific work and a series of plausible interactions in individual data visualizations, a conceptual framework for \cmvs{} is specified.
Furthermore, a formal concept of an interaction is defined, a common data model between visualizations is derived and a notification procedure is developed.
This specification is implemented for the use case of the coordination of a treemap and a geographic visualization.
Finally, common application scenarios and software-relevant requirements are evaluated and the system performance is analysed.

\begin{otherlanguage}{german}
\chapter*{Zusammenfassung}

Koordinierte, multiple Informationsvisualisierungen können einen Mehrwert bieten, indem sie die Stärken einzelner Visualisierungstechniken oder Datendarstellungen kombinieren und Benutzerinteraktionen gleichzeitig in mehreren Ansichten wiedergeben.
Insbesondere bei der Visualisierung mehrdimensionaler und hierarchischer Geodaten kann die Kombination von digitalen Karten und Treemap-basierten Visualisierungen die Verständlichkeit und Interaktivität erleichtern.
Z.B. kann eine verknüpfte Kartendarstellung den geografischen Kontext herstellen, wenn der verwendete Layoutalgorithmus der Treemap die räumlichen Daten nicht berücksichtigt.
Darüber hinaus kann der Nutzer je nach Bedarf Elemente basierend auf der geograpischen Nähe in der Karte oder basierend auf dem technischen Kontext in der Treemap auswählen.

Diese Masterarbeit beinhaltet den Entwurf und zeigt die Implementierung einer koordinierten Visualisierung einer Treemap und einer geographischen Visualisierung.
Der Fokus liegt auf koordinierten Interaktionen, den Informationen, die dabei zwischen einzelnen Ansichten ausgetauscht werden, und den gemeinsamen Schnittmengen zwischen einzelnen Datenvisualisierungen.

Aufbauend auf bestehenden wissenschaftlichen Arbeiten und einer Reihe von plausiblen Interaktionen in eigenständigen Datenvisualisierungen wird ein konzeptueller Rahmen spezifiziert.
Dabei wird ein formales Konzept einer Interaktion definiert, ein gemeinsames Datenmodell zwischen Visualisierungen abgeleitet und ein Benachrichtigungsverfahren entwickelt.
Diese Spezifikation wird für den Anwendungsfall der Koordination einer Treemap und einer geographischen Visualisierung implementiert.
Anschließend werden typische Anwendungsszenarien und software-relevante Anforderungen evaluiert und eine Performance-Analyse durchgeführt.

\end{otherlanguage}

