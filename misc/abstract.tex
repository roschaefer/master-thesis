\chapter*{Abstract}

Coordinated, multiple information visualizations can offer added value by combining the strengths of individual techniques and displaying user interactions in multiple views simultaneously.
When visualizing multi-dimensional, hierarchical and geographic data in a treemap, an additional geographic visualization may facilitate the comprehensibility.
If e.g. the used tiling algorithm places the elements of a treemap based on a non-geographical attribute, a linked geographic map can establish the geographical context.
In addition, the user can select elements as desired based on the geographic proximity in the map or based on the technical context in the tree map.

This master thesis involves the design of a coordinated visualization of a treemap and a geographic visualization.
The focus is on coordinated interactions, which information is exchanged between individual views and what the common intersection between data visualizations is.

Based on existing scientific work and a series of plausible interactions in individual data visualizations, a conceptual framework is specified.
In the process, the concept of an interaction is formalized, a common data model is defined between visualizations and a communication protocol is developed.
This specification is implemented for the use case of the coordination of a treemap and a geographic visualization.
Afterwards, the evaluation is carried out on the basis of common application scenarios, software-relevant requirements and a performance analysis.

\begin{otherlanguage}{german}
\chapter*{Zusammenfassung}

Koordinierte, multiple Informationsvisualisierungen können einen Mehrwert bieten, indem sie die Stärken einzelner Techniken kombinieren und Benutzerinteraktionen gleichzeitig in mehreren Ansichten anzeigen.
Bei der Visualisierung von mehrdimensionalen, hierarchischen und geographischen Daten in einer Treemap kann eine zusätzliche geografische Visualisierung die Verständlichkeit erleichtern.
Wenn z.B. der verwendete Tiling-Algorithmus die Elemente einer Treemap anhand eines nicht-geographischen Attributes platziert, kann eine verknüpfte Kartendarstellung den geographischen Kontext herstellen.
Darüber hinaus kann der Nutzer je nach Bedarf Elemente basierend auf der geograpischen Nähe in der Karte oder basierend auf dem technischen Kontext in der Treemap auswählen.

Diese Masterarbeit beinhaltet den Entwurf einer koordinierten Visualisierung einer Treemap und einer geographischen Visualisierung.
Der Fokus liegt auf koordinierten Interaktionen, bei denen Informationen zwischen einzelnen Ansichten ausgetauscht werden und gemeinsame Schnittmengen zwischen einzelnen Datenvisualisierungen bestehen.

Aufbauend auf bestehenden wissenschaftlichen Arbeiten und einer Reihe von plausiblen Interaktionen in eigenständigen Datenvisualisierungen wird ein konzeptueller Rahmen vorgegeben.
Dabei wird das Konzept einer Interaktion formalisiert, ein gemeinsames Datenmodell zwischen Visualisierungen definiert und ein Kommunikationsprotokoll entwickelt.
Für den Anwendungsfall der Koordination einer Treemap und einer geographischen Visualisierung wird diese Spezifikation implementiert.
Anschließend erfolgt die Auswertung anhand typischer Anwendungsszenarien, software-relevanten Anforderungen und einer Performance-Analyse.

\end{otherlanguage}

