%\documentclass[10pt,a4paper,onecolumn,twoside,openright]{book}
\RequirePackage{rotating}
\let\accentvec\vec
\documentclass[10pt,a4paper,onecolumn,twoside,openright,titlepage]{svmono}
\let\spvec\vec
\let\vec\accentvec

%\usepackage{amssymb}
\usepackage{diagbox} % for table header cell with diagonal line
\usepackage{caption}
\usepackage[english]{varioref}		% Vario REF \vref
\usepackage{hyperref}			% Clickable links
%\usepackage{breakurl}							% Umbrechende URLs
%\usepackage{url}
%\usepackage{graphics}
\usepackage{graphicx}							% Graphic support for eps
%\usepackage{ams}
\usepackage{textcomp}							% TC fonts
\usepackage{color}
%\usepackage{alltt}								% includes verbatim from text files
\usepackage{lmodern}							% Schriftanpassungen
\usepackage{multicol}							% Mehrere Spalten im Text verwenden
%\usepackage[bottom]{footmisc}			% Fußzeilen
\usepackage{pifont} 							% for fancy bullets
\usepackage{setspace}							% Zeilenabstand
\usepackage{fancyhdr}							% Kopfzeilen
\usepackage{makeidx}							% Index
%\usepackage[english,german]{babel}				%	nationale Datumsformate, etc.
\usepackage[german,english]{babel}
\usepackage{nomencl}							% Erstellt Abkürzungsverzeichnisse
%\usepackage[style=long,border=none,header=plain,cols=2,number=none,hypertoc=true,acronym=false,global=true]{glossary}   %
\usepackage{tabularx}							% Tabellen mit Type X haben automatische minipage mit Zeilenumbruch
\usepackage{tabulary}							% 
\usepackage{multirow}
\usepackage{rotating}             % Rotieren von float elementen
\usepackage{booktabs}							% Tabellen verschönern mit toprule/bottomrule
\usepackage{colortbl}							% farbige Tabellen
% \usepackage[numbers, sort]{natbib}								% Naturwissenschaftliches Zitieren

%\usepackage[small, normal, bf, up]{caption2} % Paket für schönere Bildunterschriften
%\usepackage[nottoc]{tocbibind}		% Literaturverzeichnis in Inhaltsverzeichnis aufnehmen
\usepackage{listings} % Programm-Listing
\lstset{language=C}
\usepackage{float}
\usepackage{amsmath}
\usepackage{parskip}
%\usepackage[toc,nonumberlist]{glossaries}
\usepackage[acronym,nomain]{glossaries}
\usepackage[toc]{appendix}

\usepackage{algorithmic}
\usepackage{algorithm}
\usepackage{chngcntr}
\usepackage{siunitx}
\usepackage{framed} 
\usepackage{comment}
\usepackage{tcolorbox}
\usepackage{amsfonts}
\usepackage{makecell}

\usepackage{array}






\counterwithout{footnote}{chapter}


\renewcommand*{\lstlistlistingname}{List of Listings}
% The following allows putting comments in the margin: "\marginpar{comment}"
\setlength{\marginparwidth}{3cm}
\let\oldmarginpar\marginpar
\renewcommand\marginpar[1]{\-\oldmarginpar[\raggedleft\footnotesize \textbf{#1}]{\raggedright\footnotesize \textbf{#1}}}

\newcommand{\source}[1]{\caption*{\hfill Source: {#1}} }

\fancypagestyle{plain}{										% Redefining the plain style
	\fancyhf{}
	\renewcommand{\headrulewidth}{0pt}
	\renewcommand{\footrulewidth}{0pt}
}


%Kopf- und Fußzeile
\pagestyle{fancy}																		% {fancyhdr}: besetzt Kopf- und Fusszeilen mit definiertem Inhalt
\fancyhf{}																					% clear all header and footer fields
																										% im Stil slanted-shape = geneigt; nouppercase = klein geschrieben
\fancyhead[LE,RO]{\bfseries \thepage}								% LE=left-even + RO=right-odd: \thepage=Seitenzahl
\fancyhead[LO]{\bfseries \nouppercase{\rightmark}}		% LO=left-odd: \rightmark=lower-level sectioning information
\fancyhead[RE]{\bfseries \nouppercase{\leftmark}}		% RE=right-even: \leftmark=higher-level sectioning information
\renewcommand{\headrulewidth}{0.5pt}								% Linie zum Abtrennen der Kopfzeile mit entsprechender Breite
\renewcommand{\footrulewidth}{0pt}									% Linie zum Abtrennen der Fusszeile mit entsprechender Breite (0=keine Linie)
%\setlength\headheight{13.6pt}
\headheight 13.6pt																	% wegen Schriftgröße 11pt muss die Höhe der Kopfzeile vergrößert werden

%\setacronymnamefmt{\gloshort}  % setzt den ersten Wert im Abkürzungsverzeichnis auf die Abkürzung selbst
%\setacronymdescfmt{\glolong: \glodesc}
%\makeacronym
%\newglossarystyle{modsuper}{
%  \glossarystyle{super}
%  \renewcommand{\glsgroupskip}{}
%}
\makeindex
% \makeglossaries
\newacronym{isp}{ISP}{Index Selection Problem}
\newacronym{kp}{KP}{Knapsack Problem}
\newacronym{splp}{SPLP}{Simple Plant Location Problem}
\newacronym{fptas}{FPTAS}{Fully Polynomial Time Approximation Scheme}
\newacronym{msc}{MSC}{Minimum Set Cover}
\newacronym[longplural={database administrators}]{dba}{DBA}{database administrator}
\newacronym{tco}{TCO}{total cost of ownership}
\newacronym{dbms}{DBMS}{database management system}
\newacronym{arima}{ARIMA}{Auto Regressive Integrated Moving Average}
\newacronym{tbats}{TBATS}{Trigonometric Box-Cox transform, ARMA errors, Trend and Seasonal components}
\newacronym{mae}{MAE}{Mean Absolute Error}
\newacronym{rmse}{RMSE}{Root Mean Squared Error}
\newacronym{cpu}{CPU}{Central Processing Unit}
\newacronym{mvcc}{MVCC}{MultiVersion Concurrency Control}
\newacronym{tlb}{TLB}{Translation Lookaside Buffer}
\newacronym{ooo}{OOO}{Out-Of-Order execution}
\newacronym{json}{JSON}{JavaScript Object Notation}
\newacronym{sql}{SQL}{Structured Query Language}
\newacronym{olap}{OLAP}{Online Analytical Processing}
\newacronym{oltp}{OLTP}{Online Transaction Processing}
\newacronym[longplural={Solid-State Drives}]{ssd}{SSD}{Solid-State Drive}
\newacronym{erp}{ERP}{Enterprise Resource Planning}

% um f im Mathmode weniger hässlich zu machen
\newcommand{\fmath}{\hspace{-0.1em}}
% Punkte zw. Abkürzung und Erklärung für Abkürzungsverzeichnis
\setlength{\nomlabelwidth}{.20\hsize}
\renewcommand{\nomlabel}[1]{#1 \dotfill}
%\makenomenclature


\bibliographystyle{unsrtnat} 
%\bibliographystyle{plainnat} 
\newcommand*{\Mail}[1]{\href{mailto:#1}{\protect\url{#1}}}
\newcommand*{\MailTitle}[2]{\href{mailto:#1@#2}{\protect\url{#1}}}
\newcommand{\keywords}[1]{\par\addvspace\baselineskip\noindent\keywordname\enspace\ignorespaces#1}
\newcommand{\biburl}[2]{\url{#1} (last visited #2)}
\newcommand{\itembf}[1]{\item \textbf{#1}}
\newcommand{\blankpage}{\clearpage{\pagestyle{empty}\cleardoublepage}}
\newcommand{\CC}{C\nolinebreak\hspace{-.05em}\raisebox{.4ex}{\tiny\bf +}\nolinebreak\hspace{-.10em}\raisebox{.4ex}{\tiny\bf +}}
%\def\thechapter{\Alph{chapter}}
%\def\thesection{\arabic{section}}

\onehalfspacing % {setspace}: setzt den Zeilenabstand auf 1,5
% Ränder definieren
\oddsidemargin 1in
\evensidemargin 0.5in

\renewcommand{\arraystretch}{1.5}
\renewcommand{\tabcolsep}{6pt}
\selectlanguage{english}

\definecolor{grey}{rgb}{0.85,0.85,0.85}
\definecolor{hpi_orange}{rgb}{0.88, 0.43, 0.20}
\definecolor{hpi}{rgb}{0.537,0.063,0.165}







% NON-BENNY
\usepackage{todonotes}
\usepackage{biblatex}
\usepackage{listings}
\usepackage{color}
\usepackage{copyrightbox}
\usepackage{ccicons}
\usepackage{booktabs}
\usepackage{wrapfig}
\usepackage{subfig}
\usepackage{tabulary}
\usepackage{amssymb}
\usepackage[inline]{enumitem}
\definecolor{lightgray}{rgb}{.9,.9,.9}%
\definecolor{darkgray}{rgb}{.4,.4,.4}%
\definecolor{purple}{rgb}{0.65, 0.12, 0.82}
\lstdefinelanguage{JavaScript}{%
  keywords={break, case, catch, continue, debugger, default, delete, do, else, false, finally, for, function, if, in, instanceof, new, null, return, switch, this, throw, true, try, typeof, var, void, while, with},
  morecomment=[l]{//},
  morecomment=[s]{/*}{*/},
  morestring=[b]',
  morestring=[b]",
  ndkeywords={class, export, boolean, throw, implements, import, this},
  keywordstyle=\color{blue}\bfseries,
  ndkeywordstyle=\color{darkgray}\bfseries,
  identifierstyle=\color{black},
  commentstyle=\color{purple}\ttfamily,
  stringstyle=\color{red}\ttfamily,
  sensitive=true
}

\lstset{%
  language=JavaScript,
   backgroundcolor=\color{lightgray},
   extendedchars=true,
   basicstyle=\footnotesize\ttfamily,
   showstringspaces=false,
   showspaces=false,
   numbers=left,
   numberstyle=\footnotesize,
   numbersep=9pt,
   tabsize=2,
   breaklines=true,
   showtabs=false,
   captionpos=b
 }
\bibliography{bibliography}

\newcommand{\rufu}{\textsc{Rundfunk MITBESTIMMEN}}
\newcommand{\visan}{\textsc{visual analytics platform}}
\newcommand\hmm[1]{\ifnum\ifhmode\spacefactor\else2000\fi>1000 \uppercase{#1}\else#1\fi}
\newcommand{\cmv}{\hmm{c}oordinated multiple view}
\newcommand{\cmvs}{\hmm{c}oordinated multiple views}
\newcommand{\map}{\textsc{2D} map}
\newcommand{\maps}{\textsc{2D} maps}
\newcommand{\tmap}{\textsc{2.5D} tree map}
\newcommand{\tmaps}{\textsc{2.5D} tree maps}
\newcommand{\twodTmaps}{\textsc{2D} tree maps}
\newcommand{\dss}{\hmm{d}ecision support systems}
\newcommand{\gv}{\hmm{g}eographic visualization}
\newcommand{\riso}{\texttt{RISO}}
\newcommand{\attr}[1]{\texttt{\detokenize{#1}}}
\newcommand{\gvis}{\hmm{g}eographical visualization}

\newcommand{\conceptTable}[3]{%
    \begin{center}
    {\small
        \begin{tabulary}{\textwidth}{ll}
            \bf Data structure & #1 \\

            \bf Dependent visual attributes & #2 \\

            \bf Independent visual attributes & #3  \\
        \end{tabulary}
    }
    \end{center}
}

\hyphenation{geo-gra-phic}
\hyphenation{to-po-lo-gi-cal}
