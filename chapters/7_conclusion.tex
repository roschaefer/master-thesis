\chapter{Summary and Conclusion}\label{sec:conclusion}

This thesis demonstrates how \cmvs{} can be used to improve comprehensibility and interactivity of \tmaps{}.
In particular, the approach was applied on multi-dimensional, geographical data with one additional, coordinated geographical visualization.
Apart from the specific use case, interactions in \cmvs{} were formally described and a conceptual framework was developed that can be used for arbitrary data visualizations.

Data analysts can use the new knowledge by applying \tmaps{} in the context of \dss{}.
This applies especially in application scenarios with a strong geographical context, e.g. local administration and urban planning.
It was shown that a coordinated geographical visualization can improve usability by providing orientation and a better recognition of regions and locations in \tmaps{}.
The enrichment of the original data set by adding geographical information, which is not present in the data set itself, turned out to be a great improvement.

The formalization of an interaction in \cmvs{} and the subsequent development of a conceptual framework is helpful for researchers and developers of visualization frameworks.
While \cmvs{} are a well researched topic, research was missing regarding a formalization of coordinated interactions.
Developers can built frameworks based on the conceptual framework specified in Chapter~\ref{sec:concept}.

As described in Chapter~\ref{sec:evaluation}, the framework is lightweight, has good scalability and loose coupling.
With respect to the requirements of a framework of \cmvs{} it shows good serialization and data extensibility.
But the framework is responsible only to coordinate interactions, so the main effort of implementing interactions is still there.

\section{Future Work}

In the future, the conceptual framework should be validated and tested with more data visualizations.
Chapter~\ref{sec:analysis} has a long list of examples of interactions which can be implemented for validation.

Chapter~\ref{sec:evaluation} suggests two approaches to implement reversibility of interactions.
An implementation and comparison of these approaches could be carried out.

Not in scope of this thesis but nevertheless relevant is the configuration and layout of multiple views next to each other.
This configuration could be completed with a feature to save these layouts to and load them from disk respectively.

Chapter~\ref{sec:evaluation} lists several performance issues in the handling of \attr{onmousemove} events.
The \tmap{} spends too much time during picking which is caused by an unnecessary expensive lookup of feature ids.
The performance of the \gv{} suffers on large data sets with many features, when all of them get updated.
This could be improved by selectively updating only those features, which have been changed.


\todo[inline]{Did we achieve our goals?}
\todo[inline]{Is the concept sane in regarding the implementation?}

\todo[inline]{List stuff which was not accomplished in this master thesis}

