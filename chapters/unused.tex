%\subsubsection{Use case specific problems}
%There is striking discrepancy of intentions between available metrics and the programme mandate of public broadcasting.
%TV ratings are produced by a group of 5000 representative homes equipped with a special TV box, while Radio ratings are carried out by phone surveys over the course of a year.
%Although created differently they share the same intention, i.e.\ to sell advertisements.
%
%We suspect that public broadcasting suffers from an overfitting problem:
%Since broadcasters have usage data only, they focus too much on people who still listen to the radio and watch TV.
%Young people especially show a decreasing interest in conventional mass media which has been shown in surveys.
%Public broadcasting fails in that respect that everybody has to pay and thus has a right to get access to information.

%In many cases, insights based on visualization techniques like \cmvs{} are used by experts for strategic decision-making.
%Thus, many advanced data visualizations techniques are made exclusively for professionals.
%
%On the other hand, data visualizations widely popular:
%Data journalism is one of the emerging fields in journalism.
%Facts and figures are the strongest evidence for opinionated journalistic reports.
%Not even a football match can do without a number based analysis.
%
%We believe that advanced data visualization techniques can be adopted and used in both an informative and strategic way by professionals as well as lay people.



% We use data visualizations in the context of the expenditure of public broadcasting fees in Germany.
% Since the year 2013 these fees are compulsory for every home in Germany and as a consequence, broadcasting receives €8,000,000,000 annually.
% Yet it is subject to little or no public feedback, ranking, or even debate on what constitutes value or quality.
% There is neither transparency on how the fees are spent nor a public feedback on how the fees should be spent.
%
% We see a great potential, a win-win situation to be precise, because both payers of the broadcasting fees and broadcasters themselves can benefit from each other.
% As of 2013 there is no legal opt-out anymore and people have a strong interest to say how their fees should be spent.
% Broadcasters on the other hand can evaluate their program based on the interests delivered by the users.
% This reciprocal relationship would create a public feedback for payers and a better program for broadcasters.
%
% This problem is perfectly suitable to be tackled with data visualizations.
% \todo[inline]{Explain why data visualization are good for communication}

%Our use case is creating a tool to publicly evaluate public broadcasting in Germany.

%Public broadcasting has a long history in all of Europe.
%Especially after the experiences of the Third Reich, freedom of the press and freedom of broadcasting in particular was defined in the German constitution, the basic law.
%Article 5 not only ensures the press to be free of censorship.
%It is interpreted in such a way that it guarantees broadcasting to exist and be politically and economically independent.
%It is remarkably in many ways:
%First, mass media are recognised to be a basic prerequisite for formation of opinion of the general public.
%Second, the state fosters free access to information, ie. Open Knowledge.

%Despite this highly positive ideal, reality looks somewhat different.
%Public broadcasting in Germany receives €8,000,000,000 (eight billion euros) annually, yet it is subject to little or no public feedback, ranking, or even debate on what constitutes value or quality.
%Since 2013, there is no legal opt-out for German citizen anymore.
%Every home in Germany has to pay for broadcasting whether or not the people actually use it.
%This has created numerous constitutional complaints and approximately 2 million homes in Germany refuse to pay, even it is virtually illegal to do so.

