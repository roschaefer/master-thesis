\chapter{Introduction}
The human brain processes visual information better than it processes text.
As a result, the most tangible data analyses usually come with some sort of data visualization.
On a computer, the user can interact with the data and explore different levels of granularity.
The visualization changes and the user can iteratively perform another interaction.
In many cases a great interactivity results in a great user experience.

There is a wide range of existing frameworks and implementations for data visualizations.
In some advanced cases, the data is visualized in multiple views which are linked with each other.
These \cmvs{} are interesting, as they try to make the most out of the various advantages of different visualizations.
However, implementing interactions on data visualizations, and especially on multiple visiualization, is tedious and costly.
There is currently no extensible framework that helps to implement interactions of the same data in various visualizations.

% Three moves
%\todo{What is the topic about?}
\todo[inline]{Establish the niche, why is there further research on your topic?}
\todo[inline]{Introduce the current research, what's the hypothesis, the research question?}

\section{Motivation}\label{sec:outline}

We create data visualizations of multi-dimensional, hierarchical and geographical data.
Namely, we develop \rufu{} which is an application for German citizen to publish which public broadcasts should benefit from their broadcasting fees.
The output of this application is a public user ranking and it can be used by broadcasting corporations to evaluate their program.
Visualizations of multidimensional, hierarhical data guide media researchers, journalists and the general audience to draw conclusions.

To explain this a little deeper:
Public broadcasting in Germany is organized federally, a German home belongs to the jurisdiction of a public broadcasting corporation.
But the produced content can be used by all German citizen and it is even required to be free and available for everyone.
In the application, common users vote on the entire collection of broadcasts but media researchers are only interested in users of a service area.
User accounts have a local context and broadcasts have a global context.

Let's say a media researcher is interacting with two views of the data.
The interaction may happen in one view but is reflected in another view.
E.g.\ the researcher selects user accounts from a geographical map but has the intention to update a separate view.
That view could show e.g.\ a listing of the most popular broadcasts, based on the data of user accounts from that area.

Having the interaction spread across several views creates a great value for the user but the implementation this functionality is very tedious.
We see a strong demand for \cmvs{} and the development effort as the main obstacle.
A \cmv{} composed of arbitrary charts and plots and the implementation of their interactions turns into an unsustainable amount of work.

\section{Problem Statement}
%
% \tmaps{} visualize hierarchical data on a two dimensional canvas and are particularly suitable if the proportions of the data should be emphasized.
% When dealing with both multidimensional and geographical data, problems arise when features other than geographical features are used as input of the tiling algorithm.
% This impairs the comprehensibility and complicates the selection of geographical units of items.
% A second problem is the coordination of interactions among arbitrary data visualizations.

A \tmap{} of hierarchical, geographical data can loose the geographic context if the tiling algorithm is based on non-geographical features.
Items that should belong together according to their geographic circumstances may be scattered across the \tmap{}.
This impairs the comprehensibility and complicates the selection of geographical units of items.

\section{Hypothesis}

A second, geographical visualization next to the \tmap{} can maintain the geographical context.
If an interaction in one visualization is reflected in the other, this further supports the analysis of the data.
Moreover, the limitations of a single data visualization can be avoided by interacting with the data trough another view.

%Hypothese ist, dass durchs Visualisierungs-, Navigations- und Interaktionstechniken eine Kopplung von georäumlichen 2D-Kartendarstellungen und thematisch-orientierten 3D-Treemaps ermöglicht wird und dadurch die Exploration und Analyse von multidimensionalen georäumlichen Daten unterstützt werden kann.

\section{Contributions}

In order to validate the conceptual framework for \cmvs{}, we develop a reference implementation to investigate its feasibility.
The development of the reference implementation is subdivided into the following packages:

\begin{enumerate}
  \item
    Concept

    Based on a number of examples, a basic, conceptual framework is designed.
    This includes a common data structure, a formalization of an interaction in general and the communication prototocol for multiple views.

  \item
    Implementation

    We develop a geographical represenatation of the data.
    Items in the map can be focused, highlighted, selected with multiple clicks or with a bounding box.
    This creates a powerful selection mechanism, which can be used to select data in map-based representations and highlight the data in the \tmap{}.

  \item
    Integration of \tmap{} and \map{}
    We develop linking and interaction mechanisms between \maps{} (for map-based representation) and \tmaps{} (for abstract information representation).

  \item
    Demonstration and evaluation

    \cmv{} layouts and suitable views are implemented and tested for the selected test data.
    Based on the test data sets, the \cmv{} implementation is examined and evaluated for design criteria~\cite{Baldonado2000}, general usability aspects~\cite{Roberts2007} and usage for typical visual analytics tasks.
\end{enumerate}

% \section{Methodological Approach}
%
% A literature research is used to gather knowledge about the current state of the art with respect to \cmvs{}.
% Existing concepts and implementations available on the internet are examined and reused if possible.
% Interviews of people from the target group are conducted and the define the requirements for the application.
% A minimal viable product is developed to further validate the user requirements.
% Also, common user behaviour is observed during user tests.
% The prototype is continuously developed to allow for further experimentation.
%
% \paragraph{Scenarios}
% While implementing more features, we have two scenarios with different kind of data and different user requirements:
% \begin{itemize}
%   \item
%     In our work on the project \rufu{} we design and prototype visualizations for media researchers.
%     We fully control the database and the database schema as well as on the user facing application on top of it.
%     User requirements are tied to journalists, media researchers, broadcasting corporations.
%     The data has geographical, hierarchical, temporal and correlated characteristics.
%   \item
%     The \visan{} for administrative data is used as a more general purpose application.
%     There is no single database schema but combatibility with many sources or services.
%     User requirements are potentially unknown and part of the resarch.
%     The usage focuses especially on geographical and hierarchical data.
% \end{itemize}
%
% For the scope of this master thesis we therefore compare implemented visualization and views.
% How do both approaches differ in development speed, value for the customer?
% What considerations need to be done regarding the database schema?
%
% \todo[inline]{What are the areas of research, in which this thesis can be placed into?}

\section{Structure of the Work}

In Section~\ref{sec:theory} we introduce basic terminology of \cmvs{} and the theoretical background in this area of research.
Section~\ref{sec:interaction-theory} covers the state of the art and research on multiple views and coordinated interactions.
In Section~\ref{sec:analysis} we analyze a set of data visualizations and their interactions by example.
The gained knowledge from that section is used in the following Section~\ref{sec:concept} for a formalization of the \cmv{} framework.
In Section~\ref{sec:implementation} we describe the reference implementation.
Then in Section~\ref{sec:evaluation} we take the requirements from Section~\ref{sec:analysis} to evaluate the conceptual framework and its implementation.
Finally sum up the main contribution in Section~\ref{sec:conclusion} and outline the future work in Section~\ref{sec:future-work}.

