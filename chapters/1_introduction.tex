\chapter{Introduction}
\todo[inline]{Introduction überarbeiten}
\todo[inline]{Die bestehende Implementierung erwähnen, und sagen was dazu implementiert wird}
The human brain processes visual information better than it processes text.
As a result, data scientists often communicate their results with data visualizations.
These visualizations make the data easier accessible to readers and help them to spot anomalies.

Data visualizations on a computer allow the user to interact with the data and explore different levels of granularity.
Users can use an input device like a mouse or a keyboard to trigger an interaction and the visual representation of the data will change accordingly.
In many cases a great interactivity results in a great user experience, as the user can change the visual representation as desired.

There is a wide range of existing frameworks and implementations for data visualizations.
These frameworks often support interactions with additional controls and provide an \gls{api} and event handlers for input devices.

A few specific use cases extend these frameworks to show multiple views next to each other, with controls and event handlers to manipulate multiple views.
The user experience is even better compared with single data visualizations but the development effort for multiple views is high.
Currently, there is no dedicated or prevalent framework that helps to implement interactions in multiple visualizations of the same data.

% \todo[inline]{1. Research domain}
% \todo[inline]{2. Problem statement}
% \todo[inline]{3. Lack of existing research}
% \todo[inline]{4. Own solution, contribution}
% \todo[inline]{5. What kind of validation was carried out?}

\section{Motivation}\label{sec:outline}
Many research papers in the field of \cmvs{} are more focused on visual representations than interaction aspects.
Even though interaction aspects give a great user experience and enable the user to find complex connections involving multiple views.
\textcite{Ho2013} assumes this may ``originate from the fact that the implementation of interaction techniques and interactive features normally takes much more time than the implementation of visual representations''.
There is a lot of existing research in the area but a lack of research regarding interactions in particular.
This could also explain why hardly any general-purpose implementation exists to coordinate multiple views, that focuses on interactions.
It is for this reason that this thesis develops an interaction model for \cmvs{}, so future implementations can use this model as a specification.

Another motivation to do research in that particular area is a recent development in web application frameworks:
Many popular frameworks like Angular, Ember, React and Vue have developed mechanisms to update \gls{ui} elements during user interactions.
These patterns and mechanisms have become so widespread and prevalent that they triggered even a web specification of the \gls{w3c} called ``web components''.
Obviously, these update mechanisms are a promising choice for \cmvs{}.
They have not gained research interest yet, probably because of the very recent development.

\section{Problem Statement}
\begin{tcolorbox}
\textbf{Problem Statement} \\
A treemap of geographic data may loose the geographic context if the tiling algorithm is based on non-geographic attributes.
\end{tcolorbox}

Let's take the visualization of administrative districts in Germany as an example.
An  district that is located in the east of a second district may be placed in the treemap left of it instead of being placed on the right side.
If the tiling algorithm uses a non-geographic hierarchy, the geographic context will be lost entirely.
Items that should belong together according to their geographic circumstances may be scattered across the treemap.
Users have a hard time to recognize geographic areas and locations which deteriorates the comprehensibility of the treemap.
Selecting and grouping items based on their geographic proximity becomes increasingly difficult if the items are scattered across the treemap.


\section{Hypothesis}\label{sec:introduction:hypothesis}

\begin{tcolorbox}
\textbf{Hypothesis} \\
A second, \gv{} next to the treemap can preserve the geographic context if these two views are combined in a \cmv{}.
\end{tcolorbox}

The user can relate an item in the treemap if the item is linked with the corresponding item in the \gv{} and vice versa.
Many items can be selected in the geographic map based on their proximity by dragging a bounding box around them.
In the treemap the user can select many items based on their proximity in a non-geographic dimension and see the selection in the \gv{}.
Essentially,  the limitations of a single treemap or a single \gv{} can be overcome by splitting up the interaction:
The user can trigger the interaction in one view and see the effect, i.e.\ the change of visual representation, in another view.


\section{Contributions}


\begin{minipage}{\textwidth}
	The contributions of this work are the following: \\
	\begin{tcolorbox}
		\textbf{Contributions:}
		\begin{enumerate}
  \item A formalization of interaction aspects in the field of data visualizations
  \item A conceptual framework for \cmvs{} of arbitrary data visualizations
  \item An implementation of the conceptual framework for the present use case of geographic and hierarchical data
  \item A proof of concept how treemap and \gv{} can be combined to overcome limitations of each visualization respectively
		\end{enumerate}
	\end{tcolorbox}
\end{minipage} \\



\section{Structure of the Work}

In Section~\ref{sec:related-work} the basic terminology of \cmvs{} is introduced and the theoretical background in this area of research.
This Section also covers the state of the art and research on multiple views and coordinated interactions.
In Section~\ref{sec:analysis} a set of data visualizations is analyzed and their interactions by example.
Each interaction is further examined to identify the relevant information that need to be communicated in a \cmv{}.
The gained knowledge from that section is used in the following Section~\ref{sec:concept} to develop a conceptual framework that can be used for future implementations of \cmvs{}.
This conceptual framework includes a data model shared between all views, a suggested communication protocol as well as a formalization of an interaction.
In Section~\ref{sec:implementation} the implementation of the conceptual framework is described for the present use case.
This implementation serves as reference implementation, to prove feasibility of the conceptual framework.
The implementation includes the necessary interactions to demonstrate the advantage of \cmvs{} for hierarchical and geographic data.
It also serves to validate or invalidate the hypothesis in Section~\ref{sec:introduction:hypothesis}.
In Section~\ref{sec:evaluation} the implementation is tested for typical visual analytics tasks and it is demonstrated that additional value can be generated from the combination of treemap and \gv{}.
Based on these use case scenarios, the \cmv{} implementation is examined and evaluated for design criteria~\parencite{Baldonado2000}.
A performance analysis of the implementation is carried out to demonstrate the feasibility of the conceptual framework.
The last part of the evaluation is a manual check of software requirements defined in Section~\ref{sec:analysis:requirements}.
The types of evaluation are therefore:

\begin{enumerate*}[label=(\arabic*)]
  \item Use case,
  \item evaluation according to design criteria
  \item performance profiling and
  \item manual check of software requirements.
\end{enumerate*}

Finally, the main contributions in Section~\ref{sec:conclusion} are summarized and the future work is outlined.

