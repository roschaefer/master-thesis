\chapter{Introduction}
The human brain processes visual information better than it processes text.
As a result, data scientists often communicate their results with data visualizations.
These visualizations make the data accessible to readers and help them to spot anomalies.

Data visualizations on a computer allow the user to interact with the data and explore different levels of granularity.
Users can use an input device like a mouse or a keyboard to trigger an interaction and the visual representation of the data will change accordingly.
In many cases a great interactivity results in a great user experience, as the user can change the visual representation as desired.

There is a wide range of existing frameworks and implementations for data visualizations.
These frameworks often support interactions with additional controls and event handlers like mouse click events.

A few specific use cases extend these frameworks to show multiple views next to each other, with controls and event handlers to manipulate multiple views.
The user experience is even better compared with single data visualizations but the development effort for multiple views is high.
Currently, there is no dedicated or prevalent framework that helps to implement interactions in multiple visualizations of the same data.

\todo[inline]{1. Research domain}
\todo[inline]{2. Problem statement}
\todo[inline]{3. Lack of existing research}
\todo[inline]{4. Own solution, contribution}
\todo[inline]{5. What kind of validation was carried out?}

\section{Motivation}\label{sec:outline}
Many research papers in the field of \cmvs{} have focused more on visual representations than interaction aspects.
Even though interaction aspects give a great user experience and enable the user to find complex connections involving multiple views.
\textcite{Ho2013} assumes this may ``originate from the fact that the implementation of interaction techniques and interactive features normally takes much more time than the implementation of visual representations''.
Obviously, there is a lot of existing research in the area but a lack of research regarding interactions in particular.
This could also explain why hardly any general-purpose implementation exists to coordinate multiple views, that focuses on interactions.
It is for this reason that this thesis develops an interaction model for \cmvs{}, so future implementations can use this model as a specification.

Another motivation to do research in that particular area is the recent adoption of the component pattern to build user interfaces.
Many popular frameworks like Angular, Ember, React and Vue structure a web page into several parts, so-called ``components''.
Each of these frameworks have developed update-mechanisms like ``Two way data binding'' or ``Actions up, Data down'' to create interactivity among these components.
The web component pattern has become the prevalent pattern to build user interfaces and even triggered a recent web specification with the same name.
It is a promising pattern for \cmvs{} but has not gained research interest yet, probably because of the very recent development.


% We create data visualizations of multi-dimensional, hierarchical and geographical data.
% Namely, we develop \rufu{} which is an application for German citizen to publish which public broadcasts should benefit from their broadcasting fees.
% The output of this application is a public user ranking and it can be used by broadcasting corporations to evaluate their program.
% Visualizations of multidimensional, hierarchical data guide media researchers, journalists and the general audience to draw conclusions.

% To explain this a little deeper:
% Public broadcasting in Germany is organized federally, a German home belongs to the jurisdiction of a public broadcasting corporation.
% But the produced content can be used by all German citizens and it is even free and available for everyone.
% In the application, common users vote on the entire collection of broadcasts but media researchers are only interested in users of a service area.
% User accounts have a local context and broadcasts have a global context.

% Let's say a media researcher is interacting with two views of the data.
% The interaction may happen in one view but is reflected in another view.
% E.g.\ the researcher selects user accounts from a geographical map but has the intention to update a separate view.
% That view could show e.g.\ a listing of the most popular broadcasts, based on the data of user accounts from that area.

% Having the interaction spread across several views creates a great value for the user but the implementation this functionality is very tedious.
% We see a strong demand for \cmvs{} and the development effort as the main obstacle.
% A \cmv{} composed of arbitrary charts and plots and the implementation of their interactions turns into an unsustainable amount of work.

\section{Problem Statement}
%
% \tmaps{} visualize hierarchical data on a two dimensional canvas and are particularly suitable if the proportions of the data should be emphasized.
% When dealing with both multidimensional and geographical data, problems arise when features other than geographical features are used as input of the tiling algorithm.
% This impairs the comprehensibility and complicates the selection of geographical units of items.
% A second problem is the coordination of interactions among arbitrary data visualizations.

A \tmap{} of hierarchical, geographical data can loose the geographic context if the tiling algorithm is based on non-geographical features.
Items that should belong together according to their geographic circumstances may be scattered across the \tmap{}.
This impairs the comprehensibility and complicates the selection of geographical units of items.

\section{Hypothesis}

A second, geographical visualization next to the \tmap{} can maintain the geographical context.
If an interaction in one visualization is reflected in the other, this further supports the analysis of the data.
Moreover, the limitations of a single data visualization can be avoided by interacting with the data through another view.

%Hypothese ist, dass durchs Visualisierungs-, Navigations- und Interaktionstechniken eine Kopplung von georäumlichen 2D-Kartendarstellungen und thematisch-orientierten 3D-Treemaps ermöglicht wird und dadurch die Exploration und Analyse von multidimensionalen georäumlichen Daten unterstützt werden kann.

\section{Contributions}

In order to validate the conceptual framework for \cmvs{}, we develop a reference implementation to investigate its feasibility.
The development of the reference implementation is subdivided into the following packages:

\begin{enumerate}
  \item
    Concept

    Based on a number of examples, a basic, conceptual framework is designed.
    This includes a common data structure, a formalization of an interaction in general and the communication protocol for multiple views.

  \item
    Implementation

    We develop a geographical representation of the data.
    Items in the map can be focused, highlighted, selected with multiple clicks or with a bounding box.
    This creates a powerful selection mechanism, which can be used to select data in map-based representations and highlight the data in the \tmap{}.

  \item
    Integration of \tmap{} and \map{}

    We develop linking and interaction mechanisms between \maps{} (for map-based representation) and \tmaps{} (for abstract information representation).

  \item
    Demonstration and evaluation

    \cmv{} layouts and suitable views are implemented and tested for the selected test data.
    Based on the test data sets, the \cmv{} implementation is examined and evaluated for design criteria~\cite{Baldonado2000}, general usability aspects~\cite{Roberts2007} and usage for typical visual analytics tasks.
\end{enumerate}

% \section{Methodological Approach}
%
% A literature research is used to gather knowledge about the current state of the art with respect to \cmvs{}.
% Existing concepts and implementations available on the internet are examined and reused if possible.
% Interviews of people from the target group are conducted and the define the requirements for the application.
% A minimal viable product is developed to further validate the user requirements.
% Also, common user behaviour is observed during user tests.
% The prototype is continuously developed to allow for further experimentation.
%
% \paragraph{Scenarios}
% While implementing more features, we have two scenarios with different kind of data and different user requirements:
% \begin{itemize}
%   \item
%     In our work on the project \rufu{} we design and prototype visualizations for media researchers.
%     We fully control the database and the database schema as well as on the user facing application on top of it.
%     User requirements are tied to journalists, media researchers, broadcasting corporations.
%     The data has geographical, hierarchical, temporal and correlated characteristics.
%   \item
%     The \visan{} for administrative data is used as a more general purpose application.
%     There is no single database schema but combatibility with many sources or services.
%     User requirements are potentially unknown and part of the resarch.
%     The usage focuses especially on geographical and hierarchical data.
% \end{itemize}
%
% For the scope of this master thesis we therefore compare implemented visualization and views.
% How do both approaches differ in development speed, value for the customer?
% What considerations need to be done regarding the database schema?
%
% \todo[inline]{What are the areas of research, in which this thesis can be placed into?}

\section{Structure of the Work}

In Section~\ref{sec:related-work} we introduce basic terminology of \cmvs{} and the theoretical background in this area of research.
This Section also covers the state of the art and research on multiple views and coordinated interactions.
In Section~\ref{sec:analysis} we analyze a set of data visualizations and their interactions by example.
The gained knowledge from that section is used in the following Section~\ref{sec:concept} for a formalization of the \cmv{} framework.
In Section~\ref{sec:implementation} we describe the reference implementation.
Then in Section~\ref{sec:evaluation} we take the requirements from Section~\ref{sec:analysis} to evaluate the conceptual framework and its implementation.
Finally sum up the main contributions in Section~\ref{sec:conclusion} and outline the future work.

